\chapter{Introduction}\label{chap:intro}
\section{Multicellularity: From Eukaryotic Organisms to Bacterial Biofilms}
Developmental biology studies the processes that govern the formation of patterns, shapes, and organized structures in growing multicellular organisms. These processes contribute to the development of specialized organs in animals and the formation of specialized tissues and vascular networks in plants. The synchronization of cellular growth, movement, differentiation, communication, and programmed cell death within biological systems gives rise to the multitude of unique emergent structures observed in multicellular organisms. However, eukaryotic organisms are not the only ones capable of exhibiting this type of emergent behavior.{\footnotesize\cite{Niklas2019}\cite{Lyons2015}}

Over the past several decades, accumulating evidence suggests that bacteria, once thought to be commonly unicellular organisms, very often organize into complex microbial communities known as biofilms. {\footnotesize\cite{Vlamakis2008}} These communities exhibit characteristics typically associated with multicellular organisms, such as the formation of a tissue-like consortium that displays organized spatial patterns and can self-heal after damage{\footnotesize\cite{Srinivasan2018}\cite{Wang2021}\cite{Dong2022}\cite{Dong2022_1}}. These observations have significant implications for theoretical biology. Firstly, it supports the idea that multicellularity is not exclusive to eukaryotes. Secondly, it challenges the discrete categorization that separates unicellular organisms from multicellular organisms and suggests a more nuanced and continuous transition between unicellular and multicellular structures. These findings imply that in order to improve our understanding of emergent behaviors exhibited by groups of cells, we do not need to restrict our research to multicellular eukaryotic organisms. It would be sufficient to begin by investigating the signaling networks and phenotypic differentiation in prokaryotic biofilms. These biofilms are already complex enough to exhibit significant emergent properties, including phenotype differentiation (division of labor) and cell-to-cell communication, such as paracrine signaling.{\footnotesize\cite{Lpez2009}}


\section{Insights from Reaction-Diffusion Model}

Alan Turing, a renowned mathematician and computer scientist, delved into the field of theoretical biology to explore the mechanisms by which multicellular organisms develop their shape. Turing proposed a system of equations that could explain the emergence of intricate patterns in one-dimensional and two-dimensional spaces composed of biological cells, solely through reaction and diffusion processes. Similar to Turing's reaction-diffusion model, the spatial and temporal organization of specialized bacterial functions within biofilm monolayers is hypothesized to heavily depend on two primary mechanisms: the diffusion of signaling molecules through the extracellular space between cells and the biochemical reactions occurring in the cytoplasm of the cells.

Turing's equations involve specific characteristic parameters, including diffusion constants for each substance, the number of diffusing substances, reaction rates, production rates, and decay rates. In these equations, even a slight modification to the parameters can lead to unexpected and significant impacts on the resulting macroscopic patterns. Due to the nonlinearity of this reaction-diffusion system, finding an analytical solution is practically impossible. In practice, this means that it is extremely difficult to determine the precise spatial and temporal patterns that emerge from a unique combination of these parameters. Similarly, inferring the parameters of the equations solely from observing the resulting emergent pattern is also an impossible task. Therefore, it is often necessary to rely on computational numerical simulations to obtain approximate solutions and predict these emerging behaviors. {\footnotesize\cite{turing}\cite{Landge2020}}.

Similar to Turing's reaction-diffusion system, predicting emergent behaviors in bacterial biofilms analytically can be quite challenging, even with a comprehensive understanding of the rules and parameters that govern each individual cell. We may have precise knowledge of the diffusion constant of the biofilm's signaling molecules, the rates at which they are produced, and their effects on cytoplasmic gene regulatory networks. However, due to the nonlinearities inherent in biological phenomena, it is impossible for us to analytically predict the emergent behaviors that arise from biofilms and their mechanisms.

Numerical simulations play a vital role in analyzing the complementary information that arises at different biological scales. Agent-based models, specifically, can be used to simulate the diffusion of signaling molecules across a biofilm and the individual responses of bacteria (agents) to these diffused substances. Such models can help us evaluate the comprehensiveness of our current understanding of signaling pathways and gene regulation in the formation of biofilms. Moreover, if the model can closely replicate real-world data, it eliminates the need for wet lab experimentation by predicting biofilm behavior in silico.{\footnotesize\cite{Nagarajan2022}}

\section{Objectives}
\subsection{Developing an Agent-Based Model}
The main objective of this study was to develop an agent-based computational model of \textit{Bacillus subtilis} biofilms. This involves simulating the signaling networks, phenotypic differentiation, and stochastic behavior of bacteria within a monolayer biofilm. The model was designed to simulate autonomous bacterial agents, each capable of secreting signaling molecules and carrying out internal gene regulatory circuits in response to the environmental concentration of signaling molecules. Furthermore, we have included volume exclusion interactions into the model. These interactions simulate the effects of physical constraints that limit the movement of bacterial cells.
\subsection{Visualization of Biofilm simulation}
The secondary objective was to visually represent the development of simulated monolayer biofilms. The ability to track spatio-temporal pattern formations and changes in the biofilm is crucial for understanding the overall dynamics and behavior of these systems. By utilizing graphical libraries, we created visual representations of the simulated biofilm at different stages of its development.
