\chapter*{Abstract}

Multicellularity is one of the most puzzling processes in biology. This phenomenon is driven by an intricate communication network among individual cells, which enables them to distribute tasks in an organized manner. Unfortunately, the high complexity of these signaling networks presents a significant challenge to our efforts to fully understand and predict emergent behaviors in multicellular systems.


In this thesis, we developed an agent-based model to simulate cell signaling and phenotype differentiation in \textit{Bacillus subtilis} monolayer biofilms. The model simulates bacteria as autonomous agents, including the secretion and diffusion of signaling molecules, as well as the dynamic intracellular concentrations of proteins involved in the gene regulatory network that controls matrix production and biofilm formation. The simulated biofilms exhibit an unexpected, emergent oscillatory behavior in biofilm phenotypes. These oscillations were not explicitly programmed into the agents' functions.


We believe that this model can help us improve our overall understanding of emergent functions in bacterial multicellularity, such as biofilm self-healing. It can also serve as a valuable tool in synthetic biology for designing, testing, and predicting the behavior of \textit{de novo} gene regulatory circuits and signaling networks in bacteria. The limitations, such as the computational requirements for a large number of cells and the reasonable assumptions made about the model's parameters, are also discussed.
